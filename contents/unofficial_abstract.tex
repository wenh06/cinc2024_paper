% -*- Mode:TeX -*-

\documentclass{cinc-abstract}

\usepackage{graphicx}
\usepackage{xcolor}
\usepackage{relsize}
\usepackage{pifont}
\usepackage{currfile}

\newcommand\wordcount{\input{|"texcount -inc -sum -0 -utf8 -ch -template={SUM} \currfilepath"}}


\begin{document}

% The title is set in 14-point Helvetica bold
\title{A Novel Multi-Task Learning Framework for Simultaneous Digitization and Classification of Electrocardiogram Images}

% The rest of the title block is set in 12 points Helvetica
\author {Jingsu Kang, Hao Wen\\ % First name, initials and surnames, no ``and''
\ \\ % leave an empty line between authors and affiliation
Tianjin Medical University\\  % gives affiliation of the first author only
Tianjin, China} % city, [state or province,] country only

\maketitle

%%%%%%%%%%%%%%%%%%%%%%%%%%%%%%%%%%%%%%%%%%%%%%%%%%%%%%%%%%
% NOTE: The body of the abstract (exclusive of the title, authors, and authors’ affiliations) can be up to 300 words at most
%%%%%%%%%%%%%%%%%%%%%%%%%%%%%%%%%%%%%%%%%%%%%%%%%%%%%%%%%%


Aim: This paper addresses the challenge of simultaneous digitization and classification (normal or abnormal) of electrocardiograms (ECGs) captured from images or paper printouts, as presented by the George B. Moody PhysioNet Challenge 2024. This problem is crucial in medical image analysis.

Methods: We propose an end-to-end multi-task learning framework to digitize and classify ECG images concurrently. The framework includes a shared feature extraction backbone network and two task-specific head networks. We experimented with various pretrained backbone architectures, such as ResNet, ConvNeXt, and Swin Transformer. Model training employed a combination of cross-entropy loss for classification and signal-to-noise ratio (SNR) loss for digitization. Optimization was performed using the AdamW optimizer and OneCycle learning rate scheduler. ECG image data were synthesized from the PTB-XL ECG dataset, comprising 21,799 12-lead time-series ECG recordings. We allocated 10\% of the data as a validation set for model selection, following the PTB-XL dataset's predefined split. The selection was based on a combined evaluation of classification and digitization performance.

Further study will focus on exploring and comparing an alternative several-stage solution to the digitization problem, consisting primarily of an object detection model and a self-adaptive edge sharpening algorithm for ECG curve extraction.

Results: The best submission of our team ``Revenger’’ achieved an F1 score of 0.626 (ranked 16th) for the classification task and an SNR of -13.203 (ranked 30th) for the digitization task on the hidden validation set.

Conclusion: The proposed multi-task learning framework achieved competitive performance in the Challenge, providing an effective solution to the problem of simultaneous digitization and classification of ECG images. The alternative solution is expected to improve the digitization performance further.


% \ifhandout
% % do nothing
% \else
% \begin{center}
% {\larger[2]\color{red} \ding{43}\ding{43}  Totally \wordcount words \reflectbox{\ding{43}\ding{43}}}
% \end{center}
% \fi


\end{document}
