\section{Introduction}
\label{sec:intro}

% almost finished

% Although deep learning has been gaining popularity in the field of electrocardiogram (ECG) analysis in recent years, the main focus has been primarily on digital ECG signals. However, paper printouts of ECG signals are still widely used in clinical practice, especially in developing countries. Direct analysis of ECG images and its digitization are essential for further analysis using deep learning models.

Although deep learning has gained significant popularity in the field of electrocardiogram (ECG) analysis, the primary focus has been on digital ECG signals. However, paper printouts (images) of ECG signals remain widely used in clinical practice, particularly in developing countries. The direct analysis and digitization of ECG images are crucial for further analysis using deep learning models.

% The George B. Moody PhysioNet Challenge 2024 \cite{goldberger2000physionet, cinc2024} raise the question to develop automated methods for simultaneous digitization and arrhythmia classification of ECG images, aiming to promote the development of deep learning models for ECG image analysis. In this work, we developped a multi-stage framework to address this challenge.

The George B. Moody PhysioNet Challenge 2024 \cite{goldberger2000physionet, cinc2024} has posed the challenge of developing automated methods for the simultaneous digitization and arrhythmia classification of ECG images, aiming to advance the development of deep learning models for ECG image analysis. In this work, we have developed a multi-stage framework to address this challenge.
